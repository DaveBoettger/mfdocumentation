\documentclass{article}
\usepackage{amsmath} 
\usepackage{amsfonts} 
\newcommand{\mf}{\textit{Mirrorfit}}
\newcommand{\sfont}[1]{\mathbb{#1}}
\title{\mf\ Paper}

\begin{document}
\section{Abstract}

\mf\ has been developed with point clouds measured via photogrammetry, but any technique that provides similar data, such as coordinate measuring machines or laser trackers, can also be used. 
\section{\mf\ Philosophy}

The principle goal of \mf\ is to take point cloud data representing mirror surfaces and other optical elements of a telescope and provide alignment solutions and mirror surface quality analysis.

Measured point cloud data comes in the form of a discrete set of N three-dimensional points $\sfont{P}$, with each $p_n \in \sfont{P}$ optionally having an associated error $\sigma_{n}$. 
For a typical alignment procedure the data $\sfont{P}$ will be measured in the same coordinate system but divided into M non-intersecting subsets $\sfont{E}_m \subset \sfont{P}$, each corresponding to a particular optical element of the telescope.

\mf\ needs some way of associating all points in each $\sfont{E}_m$ to a particular location on the optical element they correspond to.
This association allows an estimate of the location and orientation of the element in the coordinate system of the measurement, which is required for aligning. 

The simplest way to make this association is by measuring reference points whose relative position on each element is known in advance.
This method must be employed when elements directly corresponding to theoretically defined optical surfaces cannot have photogrammetry targets applied to them because they are delicate or inside of a cryostat or in some other way inaccessible during final alignment.
It can also be employed when optical surfaces are constructed with precise reference locations, such as tooling ball holes, at designed locations on their surface. 
However, when optical surfaces are available to be measured directly, it is usually possible to optimize mirror locations beyond what would be  
A theoretical mirror surface corresponds to a two-dimensional set of points $\sfont{M}$ located in three-dimensional space. Typically this will be a quadric surface with some given position and orientation. For each point
$m \in \sfont{M}$ we assume there is an associated surface normal vector $\hat{n}$. 

Developing an alignment solution for a telescope corresponds to finding M transformations, one for each $\sfont{E}_m$, that when applied to the points in $\sfont{E}_m$ leave all the telescope elements in their theoretical desired locations according to an optical design.


\end{document}
