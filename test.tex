\documentclass{article}
\usepackage{amsmath} 
\usepackage{amsfonts} 
\newcommand{\mf}{\textit{Mirrorfit}}
\newcommand{\sfont}[1]{\mathbb{#1}}
\title{\mf\ Paper}

\begin{document}
\section{Abstract}

\mf\ has been developed with point clouds measured via photogrammetry, but any technique that provides similar data, such as coordinate measuring machines or laser trackers, can also be used. 
\section{\mf\ Philosophy}

The principle goal of \mf\ is to take point cloud data representing mirror surfaces and other optical elements of a telescope and provide alignment solutions and mirror surface quality analysis.
We assume that telescopes are divided into distinct optical elements or surfaces whose forms and relative positions are specified by a theoretical optical design. 
Measured point cloud data come in the form of a discrete set of three-dimensional points $\sfont{P}$, with each $\bf{p_i} \in \sfont{P}$ having an associated error $\pmb{\sigma_{i}}$. 
For a typical alignment procedure the data $\sfont{P}$ will be measured in the same coordinate system but divided into non-intersecting subsets $\sfont{E}_j \subset \sfont{P}$, each corresponding to a particular element of the optical design (i.e. primary mirror, secondary mirror, receiver).

\mf\ needs a way to use all points measured in each $\sfont{E}_j$ to determine the position of the corresponding element within the measurement coordinate system.
Creating this mapping between measured points and the location of a theoretical element allows \mf\ to determine how optical elements are positioned with respect to each other in the measured data. 
Developing an alignment solution for a telescope then involves finding a set of transformations, one for each $\sfont{E}_j$, that when applied to the points in $\sfont{E}_j$ leave all the telescope elements in their theoretically desired positions and orientations.

Establishing a mapping between a set of points and a mirror's position and orientation can be challenging.
In our framework a theoretical mirror surface corresponds to a two-dimensional set of points $\sfont{M}$ located in three-dimensional space.
Typically this will be a quadric surface with some given position and orientation.
For each point $m \in \sfont{M}$ we assume there is an associated surface normal vector $\pmb{\hat{n}_m}$.
Physical mirror surfaces are constructed to approximate the theoretical surfaces but will always have some intrinsic error. 
Thus we assume that every point $m$ on a real mirror is distorted by $\delta_m$ along the theoretical surface normal at that point. 
Given a measured point $\bf{p_0}$ on a mirror we require a way to find the minimum $\delta_0$ such that for some $\bf{p_{\sfont{M}0}} \in \sfont{M}$ we can write:
$$\bf{p_0 = }\bf{p_{\sfont{M}0}} + \delta_0 \pmb{\hat{n}_0}.$$
\mf\ can then identify the optimum way to use the set of points $\sfont{E}_\sfont{M}$ to represent the mirror location by minimizing: 
$$\sum_{\bf{p_j} \in \sfont{E}_\sfont{M}} \delta_j^2.$$ 
For computational reasons this calculation is done assuming all of the $\delta$s are small and expanding to linear order in $\delta$. 
%TODO can we rewrite the last part of this paragraph to also be compatible with the HMC procedure? 


Mappings between sets of points and positions and orientations of elements can also be created by measuring reference points whose position on an element are specified in advance.
This method must be employed when elements directly corresponding to optical surfaces cannot have photogrammetry targets applied to them because they are delicate or inaccessible during final alignment.
It can also be employed when optical surfaces are constructed with precise reference points at designed locations on their surface.

However, when optical surfaces are available for direct measurement, it is usually possible to use the photogrammetry data to take into account surface imperfections and find a more optimum location.

\end{document}
